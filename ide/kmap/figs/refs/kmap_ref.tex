\documentclass{standalone}
\usepackage{karnaugh-map}
\begin{document}
\begin{karnaugh-map}[4][4][1][][]
%        \manualterms{0,0,0,0,0,0,0,0,0,0,0,0,0,0,0,0}
    \maxterms{1,3,5,7,9,10,11,12,13,14,15}
    \minterms{0,2,4,6,8}
    \implicantedge{0}{4}{2}{6}
    \implicantedge{0}{0}{8}{8}
%    \implicant{0}{}{}{8}
%    \implicantedge{0}{4}
%        \implicantcorner{4}{2}
%        \implicant{5}{15}
%%        \autoterms[X]
%		\indeterminants{2,5}        
%        \implicantcorner
%        \implicantedge{4}{12}{6}{14}
    % note: posistion for start of \draw is (0, Y) where Y is
    % the Y size(number of cells high) in this case Y=2
    \draw[color=black, ultra thin] (0, 4) --
    node [pos=0.7, above right, anchor=south west] {$XW$} % Y label
    node [pos=0.7, below left, anchor=north east] {$ZY$} % X label
    ++(135:1);
        
    \end{karnaugh-map}
%  \begin{karnaugh-map}[4][2][1][][] % note empty X and Y labels
%    \autoterms[X]
%    \implicant{4}{6}
%    \implicant{1}{5}
%  \end{karnaugh-map}
\end{document}